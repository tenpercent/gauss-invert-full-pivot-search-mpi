\documentclass[12pt]{article}
\usepackage[utf8]{inputenc}
\usepackage[russian]{babel}
\usepackage{amssymb,amsmath}
\textheight=24cm
\textwidth=16cm
\oddsidemargin=0pt
\topmargin=-3cm
\parindent=24pt
\parskip=0pt
\tolerance=2000
\flushbottom
%\def\baselinestretch{1.5} % печать с большим интервалом
\title{Обращение матрицы методом Гаусса с выбором главного элемента по строке}
\author{Подкорытов Максим}

\begin{document}

\maketitle  
% \thispagestyle{empty} % не нумеровать первую страницу

\section{Особенности MPI-реализации обращения матрицы с использованием алгоритма Гаусса с поиском главного элемента по всей матрице.}

Далее, $A$~--- это обращаемая матрица, $B$~--- вспомогательная матрица, соответствующая правой части, $G$~--- массив индексов переставляемых столбцов. Также, пусть у нас есть $p$ процессов, $s$~--- число блоков в одной блочной строке, $n$~--- длина стороны обращаемой матрицы, $m$~--- длина стороны блока. Матрицы между процессами распределяем по строкам (процессу с номером $k$ достаются блочные строки с номерами $k$, $k + p$,\dots, $k + p(\frac{s+p-1}p - 1)$). Для облегчения работы с обменом данными между процессами, выделим во всех процессах под матрицы одинаковое количество памяти ($\frac{s+p-1}pmn$), и ту часть выделенной памяти, в которой не будут лежать элементы матрицы, заполним нулями.

\subsubsection*{Основные этапы работы программы}

\subparagraph{Прямой ход алгоритма Гаусса.}
Имеется цикл длины $s$; будем считать, что в текущий момент мы находимся на шаге с номером $i$.
\begin{enumerate}
  \begin{item}
    Вычисляем первую строчку рабочей области.
  \end{item}
  \begin{item}
    Ищем главный элемент в рабочей области. После поиска делаем Allreduce, после которого каждому процессу становится известен номер процесса $rank$, в котором лежит главный блок, и координаты блока в этом процессе.
  \end{item}
  \begin{item}
    Делаем перестановку столбцов в матрице $A$; поскольку она роздана по строкам, синхронизация не нужна. Делаем перестановку индексов в массиве $G$.
  \end{item}
  \begin{item}
    Процесс с номером $rank$ умножает строку матрицы $A$, в которой содержится главный блок, на этот блок. То же самое делается с матрицей B. Затем процесс с номером $rank$ делает BCast этих строк всем процессам.
  \end{item}
  \begin{item}
    Процесс, в котором содержится $i$-я строка матриц $A$ и $B$, отсылает их процессу с номером $rank$, и записывает на их место строки, принятые при BCast на прошлом шаге.
    Процесс с номером $rank$ просто записывает на место отосланных строк принятый строки.
  \end{item}
  \begin{item}
    Все процессы вычитают из строк рабочей области (для определенности, с номером $j$) принятую строчку, домноженную на $i$-й блок в $j$-й строке слева. При этом процесс, в котором лежит строка с глобальным номером $i$, из этой строки ничего не вычитает.
  \end{item}

\end{enumerate}

\subparagraph{Обратный ход алгоритма Гаусса.}
$i$~--- текущий шаг цикла, цикл проходится от $s-1$ до $0$.
\begin{enumerate}
\begin{item}
  Вычисляем номер процесса $rank$, в котором лежит строчка с номером $i$ и локальный номер $local\_i$ этой строчки. Назовём эту строчку опорной, для определенности. Затем делается Bcast строчки матрицы $B$ с номером $local\_i$ и коренным процессом $rank$.
\end{item}
\begin{item}
  Вычисляем рабочую область~--- те строчки матрицы $B$, которые в неразбитой матрице лежат выше опорной строчки.
\end{item}
\begin{item}
  Вычитаем из строк рабочей области (для определенности, с номером $j$, который принимает все возможные значения в рабочей области) опорную строку, домноженную на $i$-й блок $j$-й строки матрицы $A$. При этом блоки опорной строки домножаются на соответствующий блок слева.
\end{item}
\end{enumerate}
\subsubsection*{Подсчет невязки}
Для экономии сил, переинициализируем матрицу так, что она будет роздана по столбцам. Тогда будет удобно посчитать каждый блок матрицы ($A^{-1}A-E$), и затем посчитать невязку.
\subsubsection*{Расчет пересылаемых данных и количества точек синхронизации}

\subparagraph{Точки синхронизации}

\begin{itemize}

\begin{item}
  После того, как найден главный блок, нужно сообщить его координаты всем процессам.
\end{item}
\begin{item}
  На каждом шаге, после окончания вычитания из своих строк, процессы начинают ждать BCast (начало нового шага).
\end{item}
\begin{item}
  Пересылка первой строки рабочей области процессу, нашедшему главный блок.
\end{item}
\begin{item}
  Пересылка строки матрицы $B$ во время обратного хода.
\end{item}
\end{itemize}
 Итого, порядка $4s$ синхронизаций.
\subparagraph{Пересылаемые данные}

\begin{itemize}

\begin{item}
  Пересылка информации о том, что главный блок найден и координат главного блока: $3\,sizeof(int)$ на каждом шаге прямого хода алгоритма Гаусса.
\end{item}

\begin{item}
 Swap строчек после поиска главного элемента: $4mn\,sizeof(double)$ на каждом шаге прямого хода алгоритма Гаусса.
\end{item}

\begin{item}
  Пересылка вычитаемой строчки при обратном ходе алгоритма Гаусса: $mn\,sizeof(double)$ на каждом шаге.
\end{item}

\end{itemize}

Получается всего $5n^{2} + o(n)$.

\end{document}
